\documentclass[tate]{jlreq}

\newcommand{\setparindent}[1]{
    \renewcommand{\jlreqparindent}{#1}
    \parindent=#1
}

\usepackage{scrextend}
\changefontsizes{1.5em}

\usepackage{kanbun}

\paperheight=35em
\paperwidth=102em
\def\vmargin{\dimexpr(\paperheight-31.001em)/2\relax}
\def\vheadsep{\dimexpr\vmargin/3\relax}
\usepackage[headsep=\vheadsep, top=\vmargin, bottom=\vmargin, left=4em, right=4em]{geometry}

\AtBeginDocument{\pagenumbering{gobble}}

\begin{document}

\setparindent{0em}
\setkanbun{tateaki=0.666, unit=1.5em, yokoaki=0, aki}

\Kanbun
 大東亞戰爭終結ノ詔書
\EndKanbun

\printkanbun

% 

\vspace{2em}

% 

\setparindent{2em}
\setkanbun{tateaki=2, unit=1em, yokoaki=2.5, aki}

\Kanbun
朕深ク鑑ミ[三]世界ノ大勢ト與[二]ニ帝國ノ現狀[一],欲シ[下]以テ[二]非常ノ措置ヲ[一]收[中]㆐拾セムト時局ヲ[上],茲ニ告ク[下]爾[二]忠良ナル[一]臣民ニ[上]。

朕ハ使[下]メタリ帝國政府ヲシテ,對シ[二]米英支蘇四國ニ[一],通[中]㆐告セ受[二]㆐諾スル其ノ共同宣言ヲ[一]旨[上]。

抑〻圖リ[二]帝國臣民ノ康寧ヲ[一],偕ニスル[二]萬邦共榮之樂ヲ[一]者,皇祖皇宗之遺範ニシテ,而朕之所[二]拳拳不ル[一レ]措カ也。曩ニ所[三]㆐以宣[二]㆐戰セル米英二國ニ[一],亦實ニ出テ[四]於庶[三]㆐幾スルニ帝國ノ自存ト與ヲ[二]東亞ノ安定[一]。如キ[三]排シ[二]他國ノ主權ヲ[一]、侵スカ[二]領土ヲ[一]者,固ヨリ非ス[三]朕カ志ニ[一]。然ルニ交戰已ニ閱シ[二]四歲ヲ[一],雖(拘)ラス[三]朕カ陸海將兵之勇戰、朕カ百僚有司之勵精、朕一億衆庶之奉公,各〻盡セルニ[二]最善ヲ[一],戰局未[二]必スシモ好轉セ[一],世界ノ大勢亦不[レ]利アラ[二]於我ニ[一]。加之敵ハ新ニ使[二]㆐用シテ殘虐ナル爆彈ヲ[一],頻ニ殺[二]㆐傷シ無辜ヲ[一],慘害ノ所[レ]及フ,真ニ至ル[レ]不ルニ[レ]可カラ[レ]測ル。而モ尚繼續セムカ[二]交戰ヲ[一],終ニ不[下]招[二]㆐來スル我カ民族之滅亡ヲ[一]而已ナラ[上],延テ可シ[三]破[二]㆐卻ス人類ノ文明ヲモ[一]。如クムハ[レ]斯ノ,朕何ヲ以テカ保シ[二]億兆ノ赤子ヲ[一],謝セムヤ[二]皇祖皇宗之神靈ニ[一]。是レ朕カ所[四]㆐以至レル[レ]使ムルニ[三]帝國政府ヲシテ應セ[二]共同宣言ニ[一]也。

朕ハ對シ[下]與[二]帝國[一]共ニ終始協[二]㆐力セル於東亞ノ解放ニ[一]之諸盟邦ニ[上],不[レ]得[レ]不ル[レ]表セ[二]遺憾之意ヲ[一]。致セハ[レ]想ヲ[中]帝國臣民ニシテ,死シ[二]於戰陣ニ[一]、殉シ[二]於職域ニ[一]、斃レタル[二]於非命ニ[一]者,及其ノ遺族ニ[上],五內為ニ裂ク。且至リテハ[下]於負ヒ[二]戰傷ヲ[一]、蒙リ[二]災禍ヲ[一]、失ヒタル[二]家業ヲ[一]者之厚生ニ[上],朕之所[二]深ク軫念スル[一]也。惟フニ今後帝國ノ可[二]㆐能キ受ク[一]之苦難ハ,固ヨリ非ズ[二]尋常ニ[一]。爾臣民之衷情モ,朕善ク知ル[レ]之ヲ。然レトモ朕ハ時運ノ所[レ]趨ク,堪ヘ[レ]難キヲ[レ]堪ヘ、忍ヒ[レ]難キヲ[レ]忍ヒ,欲ス[下]以テ為ニ[二]萬世ノ[一]開カムト[中]太平ヲ[上]。

朕ハ茲ニ得テ[三]護[二]㆐持シ國體ヲ[一],信[下]㆐倚シ爾[二]忠良ナル[一]臣民之赤誠ニ[上],常ニ與[二]爾臣民[一]共ニ在リ。若シ夫レ情之所[レ]激スル、濫ニ滋クシ[二]事端ヲ[一],或ハ如キハ[下]為ニ[三]同胞排擠、互ニ亂リ[二]時局ヲ[一],誤リ[二]大道ヲ[一],失フカ[中]信義ヲ於世界ニ[上],朕最モ戒ム[レ]之ヲ。宜シク舉國一家,子孫相傳ヘ,確ク信[二]神州ノ不滅ヲ[一],念[二]任重クシテ而道遠キヲ[一],傾ケ[二]總力ヲ於將來之建設ニ[一],篤クシ[二]道義ヲ[一],鞏クシ[二]志操ヲ[一],誓テ發[二]㆐揚シ國體ノ精華ヲ[一],可シ[レ]期ス[レ]不ラムルコトヲ[レ]後レ[二]於世界之進運ニ[一]。爾臣民,其レ克ク體セヨ[二]朕カ意[一]。
\EndKanbun

\printkanbun

% 

\vspace{2em}

% 

\setparindent{0em}

\Kanbun
裕仁
\EndKanbun

\edef\gyomei{\printkanbun}

\Kanbun
天皇
御璽
\EndKanbun

\edef\gyoji{\printkanbun}

\begin{minipage}[c]{2em}
    ~
\end{minipage}
\begin{minipage}[c]{6em}
    \setkanbun{tateaki=1, unit=2em, yokoaki=1, aki}
    \vspace*{\fill}
    \gyomei
    \vspace*{\fill}
\end{minipage}
% \fbox{
\begin{minipage}[c]{12em}
    \setkanbun{tateaki=1, unit=4em, yokoaki=1, aki}
    \gyoji
\end{minipage}
% }

\end{document}
