%% Copyright 2022 Yuanhao Chen
%
% This work may be distributed and/or modified under the
% conditions of the LaTeX Project Public License, either version 1.3
% of this license or (at your option) any later version.
% The latest version of this license is in
%   http://www.latex-project.org/lppl.txt
% and version 1.3 or later is part of all distributions of LaTeX
% version 2005/12/01 or later.
%
% This work has the LPPL maintenance status `maintained'.
% 
% The Current Maintainer of this work is Yuanhao Chen.
%
% This work consists of the files kanbun.sty, kanbun.lua,
% kanbun-example.tex and kanbun.tex.
% 
\documentclass[12pt]{ltxdockit}

\AfterTOCHead[toc]{\sffamily}
\makeatletter
  \def\@seccntformat#1{\protect\makebox[0pt][r]{\csname the#1\endcsname\hspace{\marglistsep}}}
\makeatother

\usepackage{shortvrb}
\MakeShortVerb{|}

\lstnewenvironment{example}[1][]
    {\lstset{
        basicstyle=\ttfamily,
        frame=single,
        columns=flexible,
        language=[LaTeX]TeX,
        breaklines=true,
        postbreak=\mbox{\textcolor{spot}{$\hookrightarrow$}\space},
        % morekeywords={drawuntpoint,untpoint,linkuntpoints},
        % escapeinside={<@}{@>},
    }}
    {}

\usepackage{luatexja}
    \ltjdefcharrange{10}{"2E3A}
    \ltjdefcharrange{11}{"2039-"203A}
    \ltjsetparameter{jacharrange={-1, +2, +3, -4, -5, +6, +7, -8, +9, +10, -11}}
\usepackage[match]{luatexja-fontspec}
\defaultfontfeatures{Numbers=OldStyle, Scale=MatchLowercase}
\defaultjfontfeatures{Scale=MatchLowercase}
\setmainjfont{Kozuka Mincho Pr6N}
\setsansjfont{Sarasa Mono J}
\setmonojfont[YokoFeatures={JFM=prop}]{Sarasa Mono J}
\setmainfont{Minion Pro}
\setsansfont{Myriad Pro}
\setmonofont{Sarasa Mono J}
\newfontfamily{\kanbunfamily}{Kozuka Mincho Pr6N}

\usepackage{luatexja-ruby}
    \ltjsetruby{fontcmd=\addjfontfeature{RawFeature={+ruby}}}

\usepackage{kanbun}
\newcommand{\printkanbunblock}[2]{
    {
        \parindent=0pt
        \vspace*{1\zw}
        \hfill
        \vbox{
            \hsize=#1\zw
            {
                \tate
                \addjfontfeature{Scale=1}
                #2
            }
        }
        \hfill
        \vspace*{1\zw}
    }
}

\usepackage{bxtexlogo}
\usepackage{realscripts}
\usepackage{microtype}

\usepackage{wallpaper}

\Kanbun
以テ[二]羅(ら)蝶(てふ)ヲ[一]作ルガ[二]漢文訓読ヲ[一]用(ため)ノ包(ぱつけーぢ)
\EndKanbun
\title{The \sty{kanbun} package}
\subtitle{\normalsize\normalfont\addjfontfeature{Scale=1}\printkanbun}
\author{Yuanhao Chen (\ruby[rubysmash=true]{陳|元|鎬}{ちん|げん|こう})}
\date{\scshape 15 january 2022, v1.0}


\begin{document}

\ThisCenterWallPaper{1}{kanbun-example.pdf}
\ExplSyntaxOn
\tl_set:Nn \kanbun_kumi { beta }
\ExplSyntaxOff
\maketitle
\tableofcontents


\section{Introduction}
The \sty{kanbun} package, like other \emph{kanbun-kundoku} (漢文訓読) \LaTeX{} packages (such as \sty{gckanbun}), allows users to manually input macros for elements in a \emph{kanbun-kundoku} paragraph.

More importantly, it accepts plain text input in the ``\emph{kanbun} annotation'' form when used with \LuaLaTeX, which allows typesetting \emph{kanbun-kundoku} paragraphs efficiently\footnote{The idea comes from \href{https://phesoca.com/kanbun-html/}{漢文\textsc{html}} by UntPhesoca, which is a JavaScript and \textsc{css} implementation.}.


\section{Basic example with \LuaLaTeX}
As seen in the following example, typesetting a \emph{kanbun-kundoku} paragraph with the \sty{kanbun} package requires only light annotations --- it automatically transforms the annotated plain text into \LaTeX{} macros through Lua, rather than having users type in macros themselves.

\ExplSyntaxOn
\tl_set:Nn \kanbun_tateaki { 1 }
\tl_set:Nn \kanbun_kumi { aki }
\ExplSyntaxOff
\Kanbun
月落チ烏啼キテ霜満ツ[レ]天ニ,
江楓漁火対ス[二]愁眠ニ[一]。
姑(こ)蘇(そ)城外ノ寒山寺,
夜半ノ鐘声到ル[二]客船ニ[一]。
\EndKanbun

\penalty0
\printkanbunblock{13}{\printkanbun}
\vspace{-2\zw}

\begin{example}
\documentclass{ltjtarticle}
\usepackage[kumi=aki, tateaki=1]{kanbun}
\begin{document}
\Kanbun
月落チ烏啼キテ霜満ツ[レ]天ニ,
江楓漁火対ス[二]愁眠ニ[一]。
姑(こ)蘇(そ)城外ノ寒山寺,
夜半ノ鐘声到ル[二]客船ニ[一]。
\EndKanbun
\printkanbun
\end{document}
\end{example}

Note that if you want to use this functionality, you have to run this document with \LuaLaTeX.


\section{Usage}

\subsection{Package options}
Load the package with
\begin{ltxsyntax}
    \cmditem{usepackage}\oprm{\sty{kanbun} options}|{kanbun}|
\end{ltxsyntax}

This package provides a variety of customisable features in \emph{kanbun-kundoku}.

\begin{optionlist}
    \optitem[]{scale}{\prm{ratio}}
    Sets the ratio of the size of \emph{kanji} to that of ruby texts. Default: \verb|2|.
    
    \optitem[]{fontcmd}{\prm{font command}}
    Sets the font command to use for \emph{kanji}. If \sty{luatexja-fontspec} is loaded, it is set default to \verb|\addjfontfeatures{RawFeature={+trad}}| to obtain traditional \emph{kanji} if applicable.

    \optitem[]{rubyfontcmd}{\prm{font command}}
    Sets the font command to use for ruby texts. If \sty{luatexja-fontspec} is loaded, it is set default to \verb|\addjfontfeatures{RawFeature={+ruby}}| to obtain ruby glyphs when applicable.

    \optitem[]{unit}{\prm{length}}
    Sets the base size (size of \emph{kanji}). Default: \verb|\kanbun_zw|, which is initialised as \verb|1em|.

    \optitem[]{yokoaki}{\prm{ratio}}
    Sets the horizontal space between \emph{kanji} with respect to the size of ruby texts. Default: \verb|2|. 

    \optitem[]{tateaki}{\prm{ratio}}
    Sets the vertical space between \emph{kanji} with respect to the size of ruby texts. Default: \verb|2|. 

    \optitem[]{okuriprotrusion}{\prm{ratio}}
    Sets how much \emph{okurigana} should be vertically tucked into the space of \emph{kanji} with respect to the size of ruby texts, if that does not cause an overlap with \emph{furigana}. Default: \verb|1|.

    \optitem[]{kumi}{\opt{aki}, \opt{beta}}
    Sets whether to use \emph{aki-gumi} (typeset with uniform inter-character spacing) or \emph{beta-gumi} (typeset with no inter-character space between adjacent character frames). Or simply call \opt{aki} or \opt{beta} without \opt{kumi=}. Default: \opt{aki}. 
\end{optionlist}

After initialising the options, you can still change the option values through \sty{exlp3} syntax, with a prefix \verb|\kanbun_| to option names. For example, to switch to \emph{beta-gumi}, you could use
\begin{example}
\ExplSyntaxOn
\tl_set:Nn \kanbun_kumi { beta }
\ExplSyntaxOff
\end{example}

\subsection[Basic usage without \LuaLaTeX]{Basic usage without \LuaLaTeX{} (not recommended)}
When not using the advanced \emph{kanbun}-annotation functionality, it is possible to typeset \emph{kanbun} with any engine with \textsc{cjk} support, such as using \XeLaTeX{} with the \sty{xeCJK} package, or using \upLaTeX{} with \sty{utarticle} or other appropriate class. 

\begin{ltxsyntax}
    \cmditem{kanjiunit}\verb|{ |\cmd{furiokuri}\mprm{right furigana}\mprm{right okurigana}\verb| }|\vspace{-5pt}\\
    \mprm{left punctuation (e.g.~`「')}\\
    \mprm{kanji}\\
    \mprm{other punctuation}\\
    \mprm{kaeriten}\\
    \verb|{ |\cmd{furiokuri}\mprm{left furigana}\mprm{left okurigana}\verb| }|

    \cmditem{kanbunfont}

    Sets the font size of \emph{kanji}. Use when the \opt{unit} option is set different to the document's default font size.

    Use \cmd{multifuriokuri} instead of \cmd{furiokuri} if you are putting \emph{furigana} to multiple kanji.

    \cmditem{multifuriokuri}\oprm{length by which {furigana} is raised}\mprm{furigana}\mprm{okurigana}
\end{ltxsyntax}

For example, the code
\begin{example}
% example text from https://phesoca.com/kanbun-html/
\kanbunfont
\kanjiunit{}{}{子}{}{}{}
\kanjiunit{\furiokuri{}{ク}}{}{曰}{,}{}{}
\kanjiunit{\furiokuri{}{ゾ}}{}{盍}{}{三}{\furiokuri{}{ル}}
\kanjiunit{}{}{各}{〻}{}{}
\kanjiunit{\furiokuri{}{ハ}}{}{言}{}{二}{}
\kanjiunit{\furiokuri{}{ノ}}{}{爾}{}{}{}
\kanjiunit{\furiokuri{}{ヲ}}{}{志}{。}{一}{}
\par
\end{example}
outputs

\printkanbunblock{6}{
\kanbunfont
\kanjiunit{}{}{子}{}{}{}
\kanjiunit{\furiokuri{}{ク}}{}{曰}{,}{}{}
\kanjiunit{\furiokuri{}{ゾ}}{}{盍}{}{三}{\furiokuri{}{ル}}
\kanjiunit{}{}{各}{〻}{}{}
\kanjiunit{\furiokuri{}{ハ}}{}{言}{}{二}{}
\kanjiunit{\furiokuri{}{ノ}}{}{爾}{}{}{}
\kanjiunit{\furiokuri{}{ヲ}}{}{志}{。}{一}{}
\par
}

\noindent
with \opt{tateaki} set to \verb|1|.

\subsection{Usage with \LuaLaTeX}
\emph{Kanbun} annotation uses the following brackets to mark different elements in \emph{kanbun-kundoku} (as described in \href{https://phesoca.com/kanbun-html/}{漢文\textsc{html}} by UntPhesoca). 
\begin{itemize}
    \item \verb|( )|: \emph{furigana} (振り仮名)
    \item \verb|{ }|: \emph{okurigana} (送り仮名) (these brackets can be omitted)
    \item \verb|‹ ›|: \emph{furigana} (振り仮名) of \emph{saidoku-moji} (再読文字)
    \item \verb|« »|: \emph{okurigana} (送り仮名) of \emph{saidoku-moji} (再読文字)
    \item \verb|[ ]|: \emph{kaeriten} (返り点)
    \item \verb|‘ ’|: multiple \emph{kanji}, potentially with \emph{tateten} inserted, as a ruby base; group ruby (グループルビ)
    \item no annotation: \emph{kanji} (漢字) and punctuation.
\end{itemize}
\emph{Tateten} (竪点) can be input with either \verb|―| (\verb|U+2015|), \verb|—| (\verb|U+2014|) or \verb|㆐| (\verb|U+3190|).

\begin{ltxsyntax}
    \cmditem{Kanbun}\cmditem{EndKanbun}

    Write the annotated \emph{kanbun} between the commands \cmd{Kanbun} and \cmd{EndKanbun}, and it will be processed and saved, ready to be used later.

    \cmditem{printkanbun}

    Where you would like to use the most recently saved \emph{kanbun-kundoku} paragraph, use \cmd{printkanbun}. 

    \cmditem{printkanbuncode}

    If you wish to make modifications on the result or to use the result with a non-\LuaTeX{} engine, it is possible to obtain the macros using \cmd{printkanbuncode} (prints in the terminal), and continue to work from there.
\end{ltxsyntax}

You can always save \cmd{printkanbun} to a macro and start a new annotated \emph{kanbun} block, as in the following example.
\begin{example}
% example text from https://phesoca.com/kanbun-html/
\documentclass{ltjtarticle}
\usepackage[kumi=beta]{kanbun}

\Kanbun
此レ乃チ信(しん)之‘所―[三]以’(ゆゑん)為ル[二]陛下ノ禽(とりこ)ト[一]也。
\EndKanbun
\let\信\printkanbun

\Kanbun
孤之有ルハ[二]孔明[一],猶ホ‹ごと›«キ»[二]魚之有ルガ[一レ]水也。
\EndKanbun
\let\孔明\printkanbun

\begin{document}
\孔明\par\bfseries\信
\end{document}
\end{example}

\Kanbun
此レ乃チ信(しん)之‘所―[三]以’(ゆゑん)為ル[二]陛下ノ禽(とりこ)ト[一]也。
\EndKanbun
\let\信\printkanbun

\Kanbun
孤之有ルハ[二]孔明[一],猶ホ‹ごと›«キ»[二]魚之有ルガ[一レ]水也。
\EndKanbun
\let\孔明\printkanbun

\ExplSyntaxOn
\tl_set:Nn \kanbun_kumi { beta }
\ExplSyntaxOff
\printkanbunblock{16}{\孔明\par\bfseries\信}


\end{document}
